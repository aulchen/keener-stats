\documentclass{article}
\usepackage{amsmath}
\usepackage{amsfonts}
\usepackage{amssymb}
\usepackage{bbm}


\newenvironment{proof}{\paragraph{Proof:}}{\hfill$\square$}
\newtheorem{theorem}{Theorem}
\newtheorem{lemma}[theorem]{Lemma}
\newtheorem{corollary}[theorem]{Corollary}

\newcommand{\R}{\mathbb{R}}
\newcommand{\Q}{\mathbb{Q}}
\newcommand{\Z}{\mathbb{Z}}
\newcommand{\N}{\mathbb{N}}

\newcommand{\A}{\mathcal{A}}

\author{Arthur Chen}
\title{Chapter 1 Probability and Measure}
\date{\today}

\begin{document}
\maketitle

\section*{Problem 2}

For a set $B \subset \N = \{1, 2, \dots\}$, define

\[
\mu(B) = \lim_{n \rightarrow \infty} \frac{\#[B \cap \{1, \dots n \}]}{n}
\]

when the limit exists, and let $\A$ denote the collection of all such sets.

\subsection*{Part b}

If $A$ and $B$ are disjoint sets in $\A$, show that $\mu(A \cup B) = \mu(A) + \mu(B)$.

By definition,

\begin{align*}
\mu(A \cup B) &= \lim_{n \rightarrow \infty} \frac{\#[(A \cup B) \cap \{1, \dots n \}]}{n} \\
&= \lim_{n \rightarrow \infty} \left[ \frac{\#[A \cap \{1, \dots n \}]}{n} + \frac{\#[B \cap \{1, \dots n \}]}{n}
\right] \\
&= \lim_{n \rightarrow \infty} \frac{\#[A \cap \{1, \dots n \}]}{n} + \lim_{n \rightarrow \infty} \frac{\#[B \cap \{1, \dots n \}]}{n}
\end{align*}

assuming the limits are nice enough. The second line follows because $A$ and $B$ are disjoint.

\subsection*{Part c}

Is $\mu$ a measure? Explain your answer.

No. A measure must satisfy countable additivity. Let $B_n$ be the singleton set $\{n\}$, and let $A = \cup_{n=1}^\infty B_n = \N$. It's clear that $\mu(B_n) = 0$. Then

\[
\mu(\cup_{n=1}^\infty B_n) = \mu(\N) = 1
\]

but

\[
\sum_{n=1}^\infty \mu(B_n) = 0
\]

and thus $\mu$ does not have countable additivity.

\end{document}